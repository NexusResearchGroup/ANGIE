\documentclass[12pt]{article}
\usepackage{longtable}
\usepackage{lscape}
\usepackage [round]{natbib}
\usepackage{graphicx}% Include figure files
\usepackage{hyperref}
\usepackage{multirow}
\newcommand{\keyword}[1]{\emph{\slshape #1}\index{#1}}
\bibliographystyle{apalike}
\usepackage{palatino}
\usepackage{rotating}
\usepackage{subfig}
\usepackage{lineno}
 \topmargin -1.5cm        % read Lamport p.163
 \oddsidemargin -0.04cm   % read Lamport p.163
 \evensidemargin -0.04cm  % same as oddsidemargin but for left-hand pages
 \textwidth 16.5cm
 \textheight 21.94cm 
\usepackage{color}
\usepackage {hyperref}
\hypersetup{
    bookmarks=true,         % show bookmarks bar?
    unicode=false,          % non-Latin characters in Acrobat�s bookmarks
    pdftoolbar=true,        % show Acrobat�s toolbar?
    pdfmenubar=true,        % show Acrobat�s menu?
    pdffitwindow=false,     % window fit to page when opened
    pdfstartview={FitH},    % fits the width of the page to the window
    pdftitle={My title},    % title
    pdfauthor={Author},     % author
    pdfsubject={Subject},   % subject of the document
    pdfcreator={Creator},   % creator of the document
    pdfproducer={Producer}, % producer of the document
    pdfkeywords={keywords}, % list of keywords
    pdfnewwindow=true,      % links in new window
    colorlinks=true,       % false: boxed links; true: colored links
    linkcolor=red,          % color of internal links
    citecolor=blue,        % color of links to bibliography
    filecolor=magenta,      % color of file links
    urlcolor=cyan           % color of external links
}
\usepackage{amsmath}

\begin{document}

\title{Assignment: An Agent-based Network Growth Model \footnote{Contact Arthur Huang at huang284@umn.edu for questions.}}
\maketitle

The agent-based module for this assignment can be accessed at \url{http://street.umn.edu/NetworkGrowth.html}. Scroll down for a brief description of the model on the webpage. Details about the model can be found in \citep{Huang2011net, skyway}. 

The goals of this assignment are:
\begin{itemize}
\item understanding how different incentives and disincentives influence the growth of network structure;
\item learning about the patterns of road network growth;
\item understanding how individual network builders' behavior shape the road/skyway network.
\end{itemize}

\section{Task 1: understand the model}


\subsection{The Grid-like City Scenario} 
Run the following scenario:
\begin{enumerate}
\item Select the ``Single-center grid-like city" scenario.
\item Set $\delta$ as 0.68, rounds as 30.
\item Set $Grid size$ as 7, $scale$ as 4.0, $newedgecost$ as 541, $w_{center}$ as 2900, and $w$ as 700. 
\item Click ``Go''.
\end{enumerate}

Save the output network and the graphs of network topological measures. Click "Setup" and run the scenario at least 5 times. What network patterns have emerged? What features do you have? Do you obtain different network topologies? 
\subsection{The Minneapolis Skyways Scenario}
Run the following scenario:

\begin{enumerate}
\item Select the ``Minneapolis Skyways" scenario.
\item Turn on ``show-downtown-streets''.
\item Turn on ``show-actual-skyways".
\item Set $\delta$ as 0.68, $unitedgecost$ as 320, and $unitbenefit$ as 1.4.
\item Set $rounds$ as 45.
\item Click ``Go''.
\end{enumerate}

Save the output network and the graphs of network topological measures. Click "Setup" and run the scenario at least 5 times. What network patterns have emerged? How close they are to the actual skyway network? What kind of evolutionary patterns do you observe?



\section{Task 2: understand how different parameters influence network topology}

Select the ``Single-center grid-like city'' scenario. Set $\delta$ values and $newedgecost$ as the values in Table 1 while keeping other parameters the same as Task 1. Run each arrangement and save the emerged network and the plots of the topological features.

\begin{table}
\end{table}

Discuss the following questions:
\begin{itemize}
\item What happens to the network topology when $\delta$ is the greatest and $newedgecost$ is the smallest?
\item What happens to the network topology when $\delta$ is the smallest and $newedgecost$ is the greatest?
\end{itemize} 

Select the ``Four-center grid-like city'' scenario. Set $\delta$ values and $newedgecost$ as the values in Table 1 while keeping other parameters the same as Task 1. Run each arrangement and save the emerged network and the plots of the topological features.

\begin{table}
\centering
\caption{Different parameters in the grid-like city scenario}

\begin{tabular}{ c  c   c  c c  c}
\hline \hline
& &\multicolumn{4}{c}{$\delta$} \\
\hline
& 0.05  & 0.55 & 1.0 & 1.50 & 2 \\
 \multirow{2}{*} {$newedgecost$}  & 50 & & & & \\  
&  100 & & & & \\ 
& 150 & & & & \\ 
&  200 & & & & \\ 
\hline

\end{tabular}
\end{table}

Discuss the following questions:
\begin{itemize}
\item What happens to the network topology when $\delta$ is the greatest and $newedgecost$ is the smallest?
\item What happens to the network topology when $\delta$ is the smallest and $newedgecost$ is the greatest?
\item Given the same set of parameters, how is the network in the single-center scenario different from the one in the four-center scenario? Use examples to illustrate the differences. 
\end{itemize} 

\section{Task 3: Submit a memo reporting to findings}
The recommended outline is as follows:
\begin{itemize}
\item Problem statement 
\item Methodology: describe the models and your approach in Task 1 and Task 2. Refer to \citet{Huang2011net, skyway} for details about the models.
\item Results and analysis: report findings from Task 1 and Task 2.
\item Conclusions and limitations. 
\end{itemize}

\bibliography{bib}

\end{document}